\documentclass[answers]{exam}
\usepackage[utf8x]{inputenc}
\usepackage[portuguese]{babel}
\usepackage[dvipdfmx]{graphicx}
\usepackage[a4paper, rmargin=2cm, lmargin=2cm, tmargin=2cm, bmargin=2cm]{geometry}
\usepackage{anttor}
\usepackage{fourier}
\usepackage{amssymb}

\begin{document}

\title{Problema dos Vestidos}
\author{Alexandre Rademaker}
\maketitle

Escreva senten\c{c}as em l\'ogica proposicional que
 descrevam as condi\c{c}\~oes do problemas abaixo, depois
 mostre que a solu\c{c}\~ao do  problema de fato decorre
 logicamente das senten\c{c}as (f\'ormulas em l\'ogica proposicional).

 Tr\^es irm\~as - Ana, Maria e Cl\'audia - foram a uma festa
 com vestidos de cores diferentes.  Uma vestia azul, a outra branco e
 a terceira preto. Chegando \`a festa, o anfitri\~ao perguntou quem
 era cada uma delas.  As respostas foram:
\begin{itemize}
 \item A de azul respondeu: ``Ana \'e a que est\'a de branco.''
 \item A de branco falou: ``Eu sou Maria.''
 \item A de preto disse:  ``Cl\'audia \'e quem est\'a de branco.''
\end{itemize}
  O anfitri\~ao foi capaz de identificar corretamente quem era cada pessoa considerando que:
\begin{itemize}
 \item Ana sempre diz a verdade.
 \item Maria \`as vezes diz a verdade.
 \item Cl\'audia nunca diz a verdade.
\end{itemize}
Determine a cor do vestido de cada irm\~a.

  Vamos utilizar $AA, AB$ e $AP$ para representar ``Ana veste azul'',
  ``Ana veste branco'' e ``Ana veste preto'' respectivamente; $CA, CB$
  e $CP$ para representar ``Claudia veste azul'', ``Claudia veste
  branco'' e ``Claudia veste preto'' respectivamente e $MA, MB$ e $MP$
  para representar ``Maria veste azul'', ``Maria veste branco'' e
  ``Maria veste preto'' respectivamente.

  Como cada uma das irm\~as veste algum vestido, temos:
$(AA\lor AB\lor AP)\land(CA\lor CB\lor CP)\land(MA\lor MB\lor MP)$.

Como cada irm\~a veste \textit{apenas} um vestido, temos:
$(\neg (AA\land AB)\land \neg (AB\land AP) \land \neg (AA\land AP)) \land
(\neg (CA\land CB)\land \neg (CB\land CP) \land \neg (CA\land CP)) \land
(\neg (MA\land MB)\land \neg (MB\land MP) \land \neg (MA\land
MP))$.

Como cada vestido \'e usado por alguma das irm\~as, temos:
\[
(AA\lor CA \lor MA)\land(AB\lor CB \lor MB)\land(AP\lor CP \lor MP)
\].

Como cada vestido \'e usado por \textit{apenas} uma das irm\~as, temos:
$(\neg (AA\land CA)\land \neg (AA\land MA) \land \neg (CA\land MA)) \land
(\neg (AB\land CB)\land \neg (AB\land MB) \land \neg (CB\land MB)) \land
(\neg (AP\land CP)\land \neg (AP\land MP) \land \neg (CP\land
MP))$. 

Essas senten\c{c}as podem ser separadas em f\'ormulas menores
considerando as componentes das conjun\c{c}\~oes.  Por exemplo,
podemos ver $(AA\lor AB\lor AP)\land(CA\lor CB\lor CP)\land(MA\lor
MB\lor MP)$ como o conjunto das senten\c{c}as $(AA\lor AB\lor AP),
(CA\lor CB\lor CP), (MA\lor MB\lor MP)$.

Ressaltamos que as senten\c{c}as simbolizadas acima apenas enunciam as
proposi\c{c}\~oes sobre todas as possibilidades de cada irm\~a usar
uma \'unica cor e cada cor ser usada por apenas uma irm\~a.  Vamos
agora analisar as frases ditas por cada irm\~a.

Se Ana \'e a que veste azul, ent\~ao Ana \'e a que est\'a de branco
(Ana sempre diz a verdade). Simbolizando: $AA\rightarrow AB$. 

Se Claudia for a que veste azul, ent\~ao Ana n\~ao est\'a de branco:
$CA\rightarrow \neg AB$. 

Se Ana \'e a que veste branco, ent\~ao Ana \'e Maria, o que \'e um
absurdo. Logo: $\neg AB$. 

Claudia dizendo a segunda frase n\~ao acrescenta informa\c{c}\~ao
alguma, pois sabemos que ela n\~ao \'e Maria.

Se Ana for a que veste preto, ent\~ao Claudia \'e a que veste branco:
$AP\rightarrow CB$. 

E se Claudia for a que veste preto, ent\~ao Claudia n\~ao \'e a que
veste branco: $CP\rightarrow \neg CB$, cuja informa\c{c}\~ao j\'a faz
parte do nosso conjunto de premissas.  Vamos representar todas essas
senten\c{c}as simbolizadas por $\Gamma$.  Agora basta verificarmos se
$\Gamma\models AP$, $\Gamma\models AB$, $\Gamma\models AA$,
$\Gamma\models CP$\ldots

Note que n\~ao precisamos de todas as premissas de $\Gamma$. Por
exemplo, para provarmos $\Gamma\models AP$, podemos usar apenas as
senten\c{c}as $\neg AB, AA\lor AB\lor AP$ e $AA\rightarrow AB$.

Prova por contradi\c{c}\~ao: Se $AP$ for falsa e as senten\c{c}as
$\neg AB, AA\lor AB\lor AP$ e $AA\rightarrow AB$ forem verdadeiras,
ent\~ao $AB$ \'e falsa (pois $\neg AB$ \'e verdadeira). Para que
$AA\rightarrow AB$ seja verdadeira, $AA$ tem que ser falsa.  Para que
$AA\lor AB\lor AP$ seja verdadeira, $AP$ precisa ser verdadeira, o que
\'e um absurdo, pois estamos supondo $AP$ falsa.  Logo, $AP$ n\~ao
pode ser falsa.

Resposta: Ana veste preto, Claudia branco e Maria azul.

\end{document}

%%% Local Variables:
%%% mode: latex
%%% TeX-master: t
%%% End:
